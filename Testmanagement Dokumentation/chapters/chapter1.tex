\chapter{Was ist Testmanagement?}
\section{Organisation}
\section{Planung}
\section{Erstellung}
\section{Ausf\"uhrung}
\section{Reporting}

\chapter{Ziele}
\section{Qualit\"atssteigerung}
\section{Hohe Testabdeckung}
\section{Kostensenkung}
\section{Dokumentation}

\chapter{Herausforderungen}
\section{Nicht genug Zeit zum Testen}
\section{Nicht genug Resourcen zum Testen}
\section{Testteams sind nicht immer am selben Ort}
\section{Probleme mit Anforderungen}
\section{Mit der Entwicklung synchron bleiben}
\section{Die richtigen Informationen berichten}
\section{Welche Qualit\"atsmetriken werden verwendet?}
\TODO{Beispiele!}

\chapter{Testmanagementtools}
\section{Beispiele}
\subsection{HP Quality Center}
\subsection{Testlink}
\subsection{IBM Rational Quality Manager}
\section{Kosten und Nutzen}
\subsection{Nutzen}
\begin{itemize}
	\item Standardisierung der Dokumentation (IEEE829)
	\item Verfolgbarkeit der Testartefakte
	\item Flexibilisierung der Testprozesse
	\item Optimierung/Verkürzung des kritischen Pfads
\end{itemize}
\subsection{Kosten}
\begin{itemize}
	\item Lizenzen
	\item Betrieb
	\item Training
\end{itemize}

\chapter{Empfehlungen}
\section{Starte Testaktivit\"aten fr\"uhzeitig}
\section{Iterative Tests}
\section{Wiederverwendung von Testartefakten}
\section{Verwende Anforderungstests}
\section{Nutze entfernte Testresourcen}
\section{Definieren und Durchsetzen von Flexiblen Testprozessen}
\section{Rest der Entwicklung koordinieren und einbeziehen}
\section{Kommuniziere den Status}
\section{Ziele und Ergebnise fokusieren}
\section{Automatisierung}
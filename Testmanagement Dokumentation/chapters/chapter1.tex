\chapter{Was ist Testmanagement?}
\section{Organisation}
\section{Testplanung}
Die Testplanung besch\"aftigt sich mit den Fragen \textit{Was?}, \textit{Wann?}, \textit{Wo?} und \textit{Wieso?} getestet wird.

\begin{itemize}
	\item \textit{Wieso?} - Der Grund f\"ur einen Test ist beispielsweise eine Anforderung, die validiert werden muss.
	\item \textit{Was?} - Was getestet wird ergibt aus den Anforderungen. 
	\item \textit{Wo?} - Der Ort des Tests h\"angt von den Software- und Hardwareanforderungen ab
	\item \textit{Wann?} - Wann getestet wird ergibt sich aus den Entwicklungszyklen.
\end{itemize}


\section{Testerstellung}
Bei der Testerstellung werden die n\"otigen Schritte eines Tests definiert.  

\section{Testausf\"uhrung}
\TODO{machen!}

\section{Testreporting}
Das Testreporting besch\"aftigt sich mit der Analyse und der Dokumentation der Testergebnisse.

\chapter{Ziele}
\TODO{machen!}
\section{Qualit\"atssteigerung}
\section{Hohe Testabdeckung}
\section{Kostensenkung}
\section{Dokumentation}

\chapter{Herausforderungen}
\section{Nicht genug Zeit zum Testen}
Zeit ist in der Regel ohnehin ein kritischer Faktor bei der Softwaeentwicklung. Wird die Zeit knapp, so wird meist bei der Zeit zum Testen gespart. Kommt es dann noch zu Verz\"ogerungen bei der Implementierung, so kann es durchaus sein, dass die fehlende Zeit beim Testen weg genommen wird.

\section{Nicht genug Resourcen zum Testen}
Oftmals ist es ein Problem die ben\"otigten Resourcen zum durchf\"uhren der Tests zu bekommen. Nicht nur Hardware-Resourcen k\"onnen hierbei ein Problem darstellen. Mitarbeiter, die die Tests durchf\"uhren sollen, k\"onnen ein noch gr\"oßeres Problem darstellen.

\section{R\"aumlich getrennte Teams}
R\"aumliche Trennung der Teams kann auch zu einem Problem werden. Wenn sich das Entwicklungsteam und das Teastteam beispielsweise auf verschiedenen Kontinenten mit verschiedenen Zeitzonen befinden. Dies ist durchaus g\"angig bringt allerdings technische Hindernisse mit sich. Wie werden Artefakte unter den Teams geteilt? Wie schaffen es die Teams ohne Verz\"ogerungen kordiniert zu bleiben? Wie knn die Projekteffizienze hierdurch gesteigert werden?

\section{Probleme mit Anforderungen}
Die Anforderungen zu validieren ist in der Regel die Hauptstrategie beim Testen. Damit dies auch funktionieren kann m\"ussen die Anforderungen auch testbar und vollst\"andig sein. Es muss jederzeit Zugriff auf die aktuellen Anforderungen und dem dazu geh\"origen System geben. 
\section{Mit der Entwicklung synchron bleiben}
\TODO{Umformulieren!}

\section{Die richtigen Informationen berichten}
Tests sind nur dann nützlich, wenn sie den Status der Entwicklung und deren Qualit\"at zeigen k\"onnen. Die Testergebnisse passend f\"ur alle relevante Personen zur rechten Zeit bereit zu stellen kann sich als echte Herausforderung herausstellen. 

Hierf\"ur gibt es mehrere Gr\"unde:
\begin{itemize}
	\item \textbf{Zu wenig Informationen} k\"onnen einem zu schlechten \"Uberblick \"uber das Projekt f\"uhren
	\item \textbf{Zu viele Informetionen} k\"onnen den eigentlichen Grund des Tests verdecken
\end{itemize}

Die Art der Darstellung der Ergebnisse ist auch ein wichtiger Punkt. Unabh\"angig von der Art der Darstellung(dokumentbasiert, browserbasiert, toolbasiert, etc.) ist es wichtig die Daten klar und logisch darzustellen.

\section{Welche Qualit\"atsmetriken werden verwendet?}
Eines der Zeile des Testens ist es die Quali\"at darzustellen. Hierzu m\"ussen die verwendeten Qualit\"atsmetriken festgelegt werden.

\TODO{Beispiele! evtl. Erkl\"arung}

\chapter{Testmanagementtools}
\section{Beispiele}
\subsection{HP Quality Center}
\subsection{Testlink}
\subsection{IBM Rational Quality Manager}
\section{Kosten und Nutzen}
\subsection{Nutzen}
\begin{itemize}
	\item Standardisierung der Dokumentation (IEEE829)
	\item Verfolgbarkeit der Testartefakte
	\item Flexibilisierung der Testprozesse
	\item Optimierung/Verkürzung des kritischen Pfads
\end{itemize}
\subsection{Kosten}
\begin{itemize}
	\item Lizenzen
	\item Betrieb
	\item Training
\end{itemize}

\chapter{Empfehlungen}
\section{Starte Testaktivit\"aten fr\"uhzeitig}
\section{Iterative Tests}
\section{Wiederverwendung von Testartefakten}
\section{Verwende Anforderungstests}
\section{Nutze entfernte Testresourcen}
\section{Definieren und Durchsetzen von Flexiblen Testprozessen}
\section{Rest der Entwicklung koordinieren und einbeziehen}
\section{Kommuniziere den Status}
\section{Ziele und Ergebnise fokusieren}
\section{Automatisierung}